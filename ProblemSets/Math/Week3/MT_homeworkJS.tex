 \documentclass[11pt]{article}

% Extract
% \usepackage[active,
%             generate=topology_definitions,
%             %extract-cmd={section},
%             extract-env={definition,algorithm}]{extract}

% \usepackage[active,
%             generate=topology_theorems,
%             %extract-cmd={section},
%             extract-env={theorem,corollary,claim}]{extract}

% \begin{extract*}

%%%%%%%%%%%%%%%%%%%%%%%%%%%%%%%%%%%%%%%%%%%%%%%%%%%%%%%%%%%%%%%%%%%%%%%%%%%%%%%%%%

% Packages

% AMS 
\usepackage{amsmath, amssymb, amsthm, amsbsy}
% Geometry
\usepackage{geometry}
% Colors
\usepackage[usenames,dvipsnames]{xcolor}
% Figures
\usepackage{graphicx}
\usepackage{float}
% Multi column lists
\usepackage{multicol}
% Subfigures
\usepackage{caption}
\usepackage{subcaption}
% Caligraphic
\usepackage{mathrsfs}
\usepackage{bbm}
% Bold
\usepackage{bm}
% algos
\usepackage[linesnumbered, lined, ruled]{algorithm2e}
% Spacing 
\usepackage{setspace}
% Refs/links
\usepackage[colorlinks=true, citecolor=Blue, linkcolor=blue]{hyperref}
\newcommand\myshade{85}
\colorlet{mylinkcolor}{violet}
\colorlet{mycitecolor}{PineGreen}
\colorlet{myurlcolor}{Aquamarine}

\hypersetup{
  linkcolor  = mylinkcolor!\myshade!black,
  citecolor  = mycitecolor!\myshade!black,
  urlcolor   = myurlcolor!\myshade!black,
  colorlinks = true,
}
% Bibliography
\usepackage{filecontents}
\usepackage{natbib}
% Indent
\usepackage{indentfirst}
% Pretty lists
\usepackage{enumitem}
\setlist[enumerate]{itemsep=2pt,topsep=3pt}
\setlist[itemize]{itemsep=2pt,topsep=3pt}
\setlist[enumerate,1]{label=(\roman*)}

% Code
\usepackage{listings}

% Appendix
\usepackage[toc,page]{appendix}

% Math
\usepackage{mathtools}
\usepackage{xparse}

% Equation numbering
\numberwithin{equation}{section}

% Use more than one optional parameter in a new commands
\usepackage{xargs}                      
% Todo
\usepackage[colorinlistoftodos,prependcaption,textsize=normalsize]{todonotes}
\newcommandx{\unsure}[2][1=]{\todo[linecolor=red,backgroundcolor=red!25,bordercolor=red,#1]{#2}}
\newcommandx{\change}[2][1=]{\todo[linecolor=blue,backgroundcolor=blue!25,bordercolor=blue,#1]{#2}}
\newcommandx{\info}[2][1=]{\todo[linecolor=OliveGreen,backgroundcolor=OliveGreen!25,bordercolor=OliveGreen,#1]{#2}}
\newcommandx{\improvement}[2][1=]{\todo[linecolor=Plum,backgroundcolor=Plum!25,bordercolor=Plum,#1]{#2}}
\newcommandx{\thiswillnotshow}[2][1=]{\todo[disable,#1]{#2}}

%%%%%%%%%%%%%%%%%%%%%%%%%%%%%%%%%%%%%%%%%%%%%%%%%%%%%%%%%%%%%%%%%%%%%%%%%%%%%%%%%%

% Document Settings

% Figure path
\graphicspath{{./figures/}}
% Matrix columns
\setcounter{MaxMatrixCols}{10}
% So pages will break inside long equation environments
\allowdisplaybreaks
% Font
\usepackage{mathpazo} 
\linespread{1.05}  
%\usepackage{courier}
% Geometry
\geometry{left=1in,right=1in,top=1in,bottom=1in}
% Counters
\setcounter{tocdepth}{2}
\setcounter{secnumdepth}{3}

%%%%%%%%%%%%%%%%%%%%%%%%%%%%%%%%%%%%%%%%%%%%%%%%%%%%%%%%%%%%%%%%%%%%%%%%%%%%%%%%%%

% Colors

\definecolor{Tm}{rgb}{0,0,0.80}
\newcommand{\defi}[1]{\textcolor{MidnightBlue}{\bf #1}}

%%%%%%%%%%%%%%%%%%%%%%%%%%%%%%%%%%%%%%%%%%%%%%%%%%%%%%%%%%%%%%%%%%%%%%%%%%%%%%%%%%

% Environments

\theoremstyle{plain}
\newtheorem{theorem}{\color{ForestGreen}{\textbf{Theorem}}}[section]
\newtheorem{claim}{\color{ForestGreen}{\textbf{Claim}}}[section]
\newtheorem{lemma}[theorem]{\color{ForestGreen}{\textbf{Lemma}}}
\newtheorem{proposition}[theorem]{\color{ForestGreen}{\textbf{Proposition}}}
\newtheorem{corollary}[theorem]{\color{ForestGreen}{\textbf{Corollary}}}
\newtheorem{axiom}[theorem]{\color{ForestGreen}{\textbf{Axiom}}}
\newtheorem{conjecture}[theorem]{Conjecture}
\newtheorem{case}[theorem]{Case}
\newtheorem{conclusion}[theorem]{Conclusion}
\newtheorem{criterion}[theorem]{Criterion}
\newtheorem{notation}[theorem]{Notation}
\newtheorem{problem}[theorem]{Problem}

\theoremstyle{definition}
\newtheorem{definition}{\color{MidnightBlue}{\textbf{Definition}}}[section]
\newtheorem{example}{\color{WildStrawberry}Example}[section]
\newtheorem{assumption}{Assumption}[section]
\newtheorem{calculation}{Calculation}[section]
\newtheorem{condition}[assumption]{Condition}
\newtheorem*{solution}{\color{Goldenrod}Solution}
% \newenvironment{solution}[1][\proofname]{%
%   \proof[\bf \color{Goldenrod}Solution to #1]%
% }{\endproof}

\newtheorem{exercise}{\color{YellowOrange}Exercise}[section]

% Literature Summary Standards
\newtheorem*{motivation}{Motivation}
\newtheorem*{summary}{Summary}
\newtheorem*{remark}{Remark}
\newtheorem*{model}{Model}
\newtheorem*{tresults}{Theoretical Results}
\newtheorem*{eresults}{Empirical Results}

%%%%%%%%%%%%%%%%%%%%%%%%%%%%%%%%%%%%%%%%%%%%%%%%%%%%%%%%%%%%%%%%%%%%%%%%%%%%%%%%%%

% Math macros

% Math ``brackets''
\newcommand\parens[1]{\left( #1 \right)}
\newcommand\squares[1]{\left[ #1 \right]}
\newcommand\braces[1]{\left\{ #1 \right\}}
\newcommand\angles[1]{\left\langle #1 \right\rangle}
\newcommand\ceil[1]{\left\lceil #1 \right\rceil}
\newcommand\floor[1]{\left\lfloor #1 \right\rfloor}
\newcommand\abs[1]{\left| #1 \right|}
\newcommand\dabs[1]{\left\| #1 \right\|}
\newcommand\vect[1]{\mathbf{#1}}
\newcommand\closure[1]{\overline{#1}}
\newcommand\pset[1]{\mathcal{P}\left(#1\right)}
\newcommand\inv[1]{#1^{-1}}
\newcommand\norm[1]{\lVert#1\rVert}

% inner product
\providecommand{\inner}[1]{\left\langle{#1}\right\rangle}
% stochastic dominance
\newcommand{\lesd}{\preceq_{\textrm{SD}}}

% Set builder (use \Set ultimately and separate by ;)
\DeclarePairedDelimiterX{\set}[1]{\{}{\}}{\setargs{#1}}
\NewDocumentCommand{\setargs}{>{\SplitArgument{1}{;}}m}
{\setargsaux#1}
\NewDocumentCommand{\setargsaux}{mm}
{\IfNoValueTF{#2}{#1} {#1\nonscript\:\delimsize\vert\allowbreak\nonscript\:\mathopen{}#2}}%
\def\Set{\set*}%

% Shortcut math
\newcommand{\ls}{\leqslant}
\newcommand{\gs}{\geqslant}
\def\ss{\subset}
\def\sse{\subseteq}
\def\nss{\not \ss}
\def\sps{\supset}
\def\pss{\subsetneq}
\def\prece{\preccurlyeq}
\def\condgap{\hspace{1cm}}
\def\eprec{\preceq}
% argmax and min
\newcommand{\argmax}{\operatornamewithlimits{argmax}}
\newcommand{\argmin}{\operatornamewithlimits{argmin}}
\newcommand{\es}{\emptyset}
% Implication and reverse implication
\def\imp{\Rightarrow}
\def\pmi{\Leftarrow}
% Integers up to number
\newcommand\intsfin[1]{\braces{1, \ldots, #1}}
% Logic
\def\bic{\Leftrightarrow}
% Bold and italic
\newcommand\boldit[1]{\textbf{\textit{#1}}}
% Misc math
\newcommand{\st}{\ensuremath{\ \mathrm{s.t.}\ }}
\newcommand{\setntn}[2]{ \{ #1 : #2 \} }
\newcommand{\cf}[1]{ \lstinline|#1| }
\newcommand{\fore}{\therefore \quad}
\newcommand{\tod}{\stackrel { d } {\to} }
\newcommand{\tow}{\stackrel { w } {\to} }
\newcommand{\toprob}{\stackrel { p } {\to} }
\newcommand{\toms}{\stackrel { ms } {\to} }
\newcommand{\eqdist}{\stackrel{d} {=} }
\newcommand{\iidsim}{\stackrel{\textrm{ {\sc iid }}} {\sim} }
\newcommand{\1}{\mathbbm 1}
\newcommand{\dee}{\,{\rm d}}
\newcommand{\given}{\, | \,}
\newcommand{\la}{\langle}
\newcommand{\ra}{\rangle}

% Shortcut greek
\def\a{\alpha}
\def\b{\beta}
\def\g{\gamma}
\def\D{\Delta}
\def\d{\delta}
\def\z{\zeta}
\def\k{\kappa}
\def\l{\lambda}
\def\n{\nu}
\def\r{\rho}
\def\s{\sigma}
\def\t{\tau}
\def\x{\xi}
\def\w{\omega}
\def\W{\Omega}
% Nice greek
\newcommand{\p}{\varphi}
\newcommand{\e}{\varepsilon}

% Shorcut vectors
\def\vx{\vect{x}}
\def\vy{\vect{y}}
\def\va{\vect{a}}
\def\vb{\vect{b}}

\newcommand{\CC}{\mathbb C}
\newcommand{\FF}{\mathbb F}
\newcommand{\RR}{\mathbb R}
\newcommand{\NN}{\mathbb N}
\newcommand{\PP}{\mathbbm P}
\newcommand{\EE}{\mathbbm E}
\newcommand{\TT}{\mathbbm T}
\newcommand{\VV}{\mathbbm V}
\newcommand{\QQ}{\mathbb Q}
\newcommand{\WW}{\mathbbm W}
\newcommand{\ZZ}{\mathbbm Z}
\renewcommand{\SS}{\mathbbm S}

% Expectation/Probability
\newcommand{\ee}[1]{\mathbbm{E}[{#1}]}
\newcommand{\pp}[1]{\mathbbm{P}({#1})}

\newcommand{\GG}{\mathsf G}
\newcommand{\XX}{\mathsf X}
\renewcommand{\AA}{\mathsf A}
\newcommand{\YY}{\mathsf Y}
\newcommand{\ZZZ}{\mathsf Z}

\newcommand{\aA}{\mathscr A}
\newcommand{\iI}{\mathscr I}
\newcommand{\eE}{\mathscr E}
\newcommand{\fF}{\mathscr F}
\newcommand{\rR}{\mathscr R}
\newcommand{\lL}{\mathscr L}
\newcommand{\cG}{\mathscr G}

\newcommand{\pP}{\mathcal P}
\newcommand{\aAA}{\mathcal A}
\newcommand{\vV}{\mathcal V}
\newcommand{\mM}{\mathcal M}
\newcommand{\oO}{\mathcal O}
\newcommand{\gG}{\mathcal G}
\newcommand{\hH}{\mathcal H}
\newcommand{\tT}{\mathcal T}
\newcommand{\bB}{\mathcal B}
\newcommand{\zZ}{\mathcal Z}
\newcommand{\cC}{\mathcal C}
\newcommand{\dD}{\mathcal D}
\newcommand{\wW}{\mathcal W}
\newcommand{\uU}{\mathcal U}
\newcommand{\sS}{\mathcal S}

% Common collections
\def\cA{\col{A}}
\def\cB{\col{B}}
% \def\cC{\col{C}}
\def\cT{\col{T}}
\def\cU{\col{U}}

% Common closures
\def\clA{\closure{A}}
\def\clB{\closure{B}}
\def\clK{\closure{K}}

% operators
\DeclareMathOperator{\cl}{cl}
\DeclareMathOperator{\graph}{graph}
\DeclareMathOperator{\interior}{int}
\DeclareMathOperator{\Prob}{Prob}
\DeclareMathOperator{\determinant}{det}
\DeclareMathOperator{\trace}{trace}
\DeclareMathOperator{\sgn}{sgn}
\DeclareMathOperator{\Span}{span}
\DeclareMathOperator{\diag}{diag}
\DeclareMathOperator{\proj}{proj}
\DeclareMathOperator{\rank}{rank}
\DeclareMathOperator{\cov}{Cov}
\DeclareMathOperator{\corr}{Corr}
\DeclareMathOperator{\var}{Var}
\DeclareMathOperator{\mse}{mse}
\DeclareMathOperator{\se}{se}
\DeclareMathOperator{\row}{row}
\DeclareMathOperator{\col}{col}
\DeclareMathOperator{\range}{rng}
\DeclareMathOperator{\kernel}{ker}
\DeclareMathOperator{\dimension}{dim}
\DeclareMathOperator{\bias}{bias}
\DeclareMathOperator{\dom}{dom}
\DeclareMathOperator{\ran}{ran}
\DeclareMathOperator{\Int}{Int}
\DeclareMathOperator{\Cl}{Cl}
\DeclareMathOperator{\im}{im}
\DeclareMathOperator{\conv}{conv}

% \end{extract*}


\title{Math - Problem Set 1 : Measure Theory}
\author{Jeanne Sorin}


\begin{document}
\maketitle

\section*{Section 1} % (fold)
\label{sec:section_1}

\subsection*{Exercise 1.3} % (fold)

\noindent $\mathcal{G}_{1}=\{A : A \subset \mathbb{R}, A \text { open }\}$: G1 is the set of open sets in R and is a $\s$ algebra because:
\begin{itemize}
	\item G1 is a non-empty set.
	\item G1 includes the empty set because it is an open set.
	\item G1 includes the complement of the empty set, R, because it is an open set in R : G1 is therefore closed under complements.
	\item The union of open sets is also open, so G1 is closed under finite unions.
	\item G1 is closed under countable unions: when n goes to infinity, the union of $A_n$ is also in A because countable unions of open sets are also open.
\end{itemize}



\noindent $\mathcal{G}_{2}=\{A : A \text { is a finite union of intervals of the form }(\mathrm{a}, \mathrm{b}],(-\infty, \mathrm{b}], \text { and }(\mathrm{a}, \infty)\}$ is an algebra but not a $\s$ algebra:
\begin{itemize}
	\item G2 is a non-empty set.
	\item G2 includes the empty set (for a = b) and its complement R (for $(-\inf , \infty)$.
	\item G2 includes intervals and their complements, as well as finite unions of intervals by construction.
	\item G2 does not include countable unions and is therefore not a $\s$ algebra.
\end{itemize}



\noindent $\mathcal{G}_{3}=\{A : A \text { is a countable union of intervals of the form }(\mathrm{a}, \mathrm{b}],(-\infty, \mathrm{b}], \text { and }(\mathrm{a}, \infty)\}$ is a $\s$ algebra:
\begin{itemize}
	\item G3 is a non-empty set.
	\item G3 includes the empty set (for a = b) and its complement R.
	\item G3 includes intervals and their complements, as well as finite unions of intervals by constructions.
	\item G3 is closed under countable unions by construction.
\end{itemize}




\subsection*{Exercise 1.7} % (fold)
\noindent Let $A = \left\{ \emptyset, X \right\}$ be the smallest $\s$ algebra on X. 
Suppose $\exists S$ a $\s$ algebra on X st ($\emptyset, X$) $\notin$ S. If the empty set is not in S then we have a contradiction. If X $\notin$ in S then $\emptyset^c \notin S$ then we have a contradiction. Therefore $(\emptyset, X)$ is included in S.
Suppose $\exists S'$ the smallest $\s algebra$ on X st X $\notin (\emptyset, X)$, then S' there are some elements of S that are not included in X. S' therefore has at least 4 elements $(\emptyset, X, T, T^c)$ but S' is larger than S and we have a contradiction.
\\
\\
\noindent 
Suppose $\exists$ B, a $\s$ algebra on X st $B \nss \pP(X)$. Then $\exists$ L $ \in B \st L \notin \pP(X)$, ie $L \notin X$ which is a contradiction. Therefore $A = \left\{P(X) \right\} = \left\{A:A in X \right\}$ is the largest $\s$ algebra on X because in contains all other $\s$ algebra on X.




\subsection*{Exercise 1.10} % (fold)

Let $\braces{S_\a}$ be a family of $\s$-algebras on X. 
All $S_\a$ contain $\emptyset$ so the $\emptyset$ is also included in their interesections. 
Let A an element in the intersection of all $S_\a$: if A is in $S_\a$, then $A^c$ is too and $A^c$ is also included in the intersection and the family of $\s$ algebra on X is therefore closed under complements. 
$S^\a$ are $\s$ algebra so they are closed under countable unions. Their intersection is therefore also closed under countable unions (and under finite unions) and is a $\s$ algebra.



\subsection*{Exercise 1.22} % (fold)

\noindent $\mu$ is a measure on (X,s) meaning that $\mu(\emptyset)=0$ and $\mu(UA_i) = \sum \mu(A_i)$ for $A_i \cap A_j = \emptyset$. 
\\
\\
$\mu$ is monotone:\\ 
Let $A \ss B the B = A \cup D for D \ss B and D \cap A = \emptyset$. Then $\mu(B) = \mu(A \cup D) = \mu(A) + \mu(D)$ by countable additivity. and $\mu(A) + \mu(D) \geq \mu(A)$ because $\mu(D) \geq 0$. 
\\
\\
$\mu$ is countably subbaditive: Let's take $A_i$ and $A_j$ in A:
\\
$\mu(A_i \cup A_j) = \mu((A_i \cap A_j^c) \cup (A_i \ss \cap A_j) \cup (A_i \cap A_j)) = \mu(A_i \cap A_j^c) \cup \mu(A_i$c$ \cap A_j) \cup \mu(A_i \cap A_j)$ by countable additivity. \\
If $A_i \cap A_j = \emptyset$ then $\mu(A_i \cap A_j) = \emptyset$ and $\mu(A_i \cup A_j) = \mu(A_i) + \mu(A_j)$. 
\\
If $A_i \cap A_j \neq \emptyset$ then $\mu(A_j \cap A_i) > \emptyset$ and $\mu(A_i \cup A_j) <= \mu(A_i) + \mu(A_j)$. 



\subsection*{Exercise 1.23} % (fold)


Let $A,B \ \in \ S $ such that $\l (A) = \mu(A\cap B)$
\begin{itemize}
	\item $\l (\emptyset) = \mu( \emptyset \cap B) = 0 $
	\item Let $\braces{A_i, A_j}$ two elements of a family of disjoint sets in S, without loss of generality:
	\begin{align*}
	\l (A_i \cup A_j) &= \mu((A_i \cup A_j) \cap B)\\
					&= \mu(A_i \cup A_j \cap B)\\
					&= \mu((A_i \cap B) \cup (A_j \cap B)) \\
					&= \mu(A_j \cap B) + \mu(A_i \cap B)\\
					&= \l(A_j) + \l(A_i)
	\end{align*}
	\noindent So $\l$ is also countably additive.
	
\end{itemize}


\subsection*{Exercise 1.26} % (fold)

Let $\mu$ a measure on (X,S). Then $\mu$ is continuous from above: \footnote{Thanks Thomas' Pellet for debugging my $\cap$ / $\cup$ / $^c$ logical issues.}\\ 
\begin{align*}
	\mu (\cap^\infty_{i=1} A_i) &= \mu \parens{\parens{\cup^\infty_1 A^c_i}^c} \\
						   &=  \mu (X) - \mu \parens{\cup^\infty_1 A^c_i}\\
						   &= \mu (X) - \lim_{n \to \infty} \mu \parens{A^c_n} \quad \text{using} \ i) \\
						   &= \mu (X) - \mu (X) +\lim_{n \to \infty} \mu \parens{A_n} \\
						   &=  \lim_{n \to \infty} \mu \parens{A_n}
\end{align*}




\section*{Section 2} % (fold)
\label{sec:section_2}

\subsection*{Exercise 2.10} % (fold)

\begin{align*}
	\mu^{*}(B) &= \mu^{*}(B \cap E \cup E \cap E^c) \subseteq \mu^{*}(B \cap E) + \mu^{*}(B \cap E^c)
\end{align*}
By subadditivity of disjoint elements:
\begin{align*}
	\mu^{*}(B) = \mu^{*}(B \cap E) + \mu^{*}(B \cap E^c)
\end{align*}



\subsection*{Exercise 2.14} % (fold)
Prove that $\bB (\RR)$ is a subset of A, i.e. that $ \s (\oO) = \s (\AA)$ using the Cathéodory extension: By definition $ \mathcal{A}=\{A : A \text { is a finite disjoint union of intervals } (\mathrm{a}, \mathrm{b}],(-\infty, \mathrm{b}], \text { and }(\mathrm{a}, \infty) \}$. The $\s$ algebra generated by A contains all open and closed sets in R, and in closed under complements, so $\bB (\RR) \in \sigma(A)$.
%\\
%Moreover, let $B \ss B(\RR)$. B is covered by an open set st $B \in (a,b)$ so $B \ss \sigma(A)$



\section*{Section 3} % (fold)
\label{sec:section_3}

\subsection*{Exercise 3.1} % (fold)

A countable subset of the real line is a singleton because each element of the real line is surrounded by irrational numbers.
Let $(x) in R$ a singleton set. $x$ is a countable set (of size 1). Suppose that $\exists \e$ st $\mu(x) > \e$ according to $\mu$ the Lebesque measure.
By construction, $(x) in (x - \frac{\e}{2} ; x + \frac{\e}{2})$.
By use subadditivity (because $\mu$ is a measure) and get:
\begin{align*}
	\mu(x) &\leq \mu((x-\frac{\e}{2}, x+\frac{\e}{2})) \leq \e
\end{align*}
This contradicts the initial supposition, even for $\lim_{O} \e$. Therefore the measure of a countable subset of the real line, a singleton is 0.




\subsection*{Exercise 3.7} % (fold)
One can define a function $g : X \to R$ measurable st $g(x) = -f(x)$. For any a:
$\braces{x \in X : f(x) <a} = \braces{x \in X : -g(x) >a}$.
\\
$f(x) < a $ corresponds to $f(x) \ss (-\inf, a) $ so the complement of this set is also in M (because M is a $\s$ algebra).
\\
All sets $f^{-1}(-\inf, a] ; f^{-1}(-\inf, a) ; f^{-1}(a, \inf) , f^{-1}[a, \inf)]$ are also on M.

\subsection*{Exercise 3.10}

See lecture notes for the proof.


\subsection*{Exercise 3.17}

Suppose that f is bounded. $\exists c \ st,\forall x \in X \quad f(x) \leq c$
Given that f is bounded by c.
\begin{align}
\abs{s_c(x) - f(x)} &= \abs{\sum_{i=1}^{c \cdot 2^{c}} \frac{i-1}{2^{c}} \chi_{E_{i}^{c}}+ c \chi_{E_{\infty}^{c}} - f(x)} \\
					&= \abs{\sum_{i=1}^{c \cdot 2^{c}} \frac{i-1}{2^{c}} \chi_{E_{i}^{c}} - f(x)_{|f(x)<c}}
\end{align}


\noindent Note that for any x, since $f(x)$ is bounded $f(x) \in [0,c]$ so $x \in E_{i}^{c}$ for some i. Note there is an $N \geq c$ such that $\frac{1}{2^{N}}< \epsilon$. 
\\
 Then for any $n \geq N$, $\left|f(x)-s_{n}(x)\right|<\epsilon$, so we have uniforme convergence.
 \\
 \\
 In other words: for $x<c$ then $c \chi_{E_{i}^{c}} - f(x) = 0$ because $f(x)=c$ and one can get rid of the extra term in the equation. 
 We reach the definition of uniform convergence as this inequality ($\abs{s_c(x) - f(x)} < \epsilon$) must hold for any $\epsilon$, even arbirately small, for each x.

\section*{Lebesgue Convergence}

\subsection*{Exercise 4.13}
Because $|f|<M \text { on } E \in \mathcal{M}$ and $\mu(E)<\infty$ we have using 4.7:
\begin{align}
\int_{E} f^+ d \mu &< \int_{E} M \chi_{E_{M}} d \mu = M \mu (E_M) < \infty \\
\int_{E} f^- d \mu &< \int_{E} M \chi_{E_{M}} d \mu = M \mu (E_M) < \infty
\end{align}

F is therefore integrable and $f \in \mathscr{L}^{1}(\mu, E)$.

\subsection*{Exercise 4.14}

Suppose $f \in \mathscr{L}^{1}(\mu, E)$. \\
Let $ E_n = \braces{x \in X: f(x) \geq n}$ and $E = \cup^{\infty}_{n=1} E_n$ Using the properties of the integral, we have that:

\begin{align}
\int_{X} f d \mu &\geq \int_{E_n} f d \mu \geq \int_{E_n} n \chi_{E_{n}} d \mu = n \mu(E_n) \geq n \mu(E) \\
\imp \mu(E) &\leq \int_{X} \frac{1}{n} f d \mu \sim \oO (\frac{1}{n})
\end{align}
This tends to zero as $n \to \infty$



\subsection*{Exercise 4.15}



Suppose $f,g \in \mathscr{L}^{1}(\mu, E)$ and $f \leq g$: 
\begin{align}
\braces{\int_{E} s d \mu: 0 \leq s \leq f, \text{s simple}} &\ss \braces{\int_{E} s d \mu: 0 \leq s \leq g, \text{s simple}}
\end{align}
Taking the supremum and by definition of the integral, we have that

\begin{align}
\int_{E} f d \mu \leq \int_{E} g d\mu
\end{align}

\subsection*{Exercise 4.16}
Suppose $f \in \mathscr{L}^{1}(\mu, E)$ and $A \ss E$
we have that: 


\begin{align}
\int_{A} f^+ d \mu &\leq \int_{E} f^+ d d \mu < \infty \\
\int_{A} f^- d \mu &leq \int_{E} f^- d \mu  < \infty
\end{align}


and therefore $f \in \mathscr{L}^{1}(\mu, A)$

\subsection*{Exercise 4.21}
If $A, B \in \mathcal{M}, B \subset A \text { and } \mu(A-B)=0, \text { then if } f \in \mathscr{L}^{1}$:

\begin{align}
\int_{A} f^+ d \mu - \int_{B} f^+ d \mu &= \int_{A\cap B \cup A \cap B^C} f^+ d \mu - \int_{B} f^+ d  \\
 		&= \int_{B} f^+ d \int_{A \cap B^C} f^+ d \mu - \int_{B} f^+ d \\
		& = \int_{A \cap B^C} f^+ d \mu \leq c \mu (A \cap B^C) = 0
\end{align}
The two integrals are therefore identical. The same is true for $f^-$.





\end{document}